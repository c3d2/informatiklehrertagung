\documentclass[a4paper]{scrartcl}
\usepackage[utf8]{inputenc}
\usepackage[ngermanb]{babel}
\usepackage{ifpdf}
\usepackage{hyperref}
\title{Appendix Folien "Umgang mit Sozialen Netzwerken"}
\author{Paul Schwanse}
\date{23. März 2012}
\ifpdf
\hypersetup{
    pdfauthor={Paul Schwanse},
    pdftitle={Appendix Folien "Umgang mit Sozialen Netzwerken"},
}
\fi
\begin{document}

Folie 3:
	\begin{itemize}
\item Prof. Pfitzmann, Sprechzettel BVerfG Onlinedurchsuchung \url{http://dud.inf.tu-dresden.de/literatur/BVG2007-10-10.pdf}
\end{itemize}


Folie 4:
\begin{itemize}
\item \url{http://c3d2.de}
\item \url{http://ccc.de}
\item \url{http://datenspuren.de}
\item \url{http://ccc.de/schule}
\item \url{http://c3d2.de/schule.html}
\end{itemize}


Folie 5:
\begin{itemize}
\item Informationelle Selbstbestimmung \url{https://secure.wikimedia.org/wikipedia/de/wiki/Informationelle_Selbstbestimmung}
\end{itemize}


Folie 7:
\begin{itemize}
\item Internet \url{https://secure.wikimedia.org/wikipedia/de/wiki/Internet}
\end{itemize}


Folie 10:
\begin{itemize}
\item TODO: Vervollständigung Bild mit Beispielnachrichten
\end{itemize}


Folie 13:
\begin{itemize}
\item Jabber \url{https://secure.wikimedia.org/wikipedia/de/wiki/Jabber_Instant_Messenger}
\item Diaspora \url{https://secure.wikimedia.org/wikipedia/de/wiki/Diaspora_(Software)}
\item Identi.ca \url{https://secure.wikimedia.org/wikipedia/de/wiki/Identi.ca}
\end{itemize}

Der C3D2 stellt, neben vielen anderen Anbietern, einen Jabber-Service (@hq.c3d2.de) so wie einen Diaspora-Pod (\url http://pod.hq.c3d2.de) zur Verfügung. Gerne sind wir Schulen dabei behilflich, ihre eigene Infrastruktur aufzusetzen und stehen für Fragen zur Verfügung.


Folie 17:
\begin{itemize}
\item vgl. JIM Studie 2011, S. 53f. (\url{http://mpfs.de/index.php?id=11})
\end{itemize}


Folie 18:
\begin{itemize}
\item Buddycloud \url{http://buddycloud.com/}
\end{itemize}


Folie 20:
\begin{itemize}
\item Jamendo \url{http://jamendo.com}
\item Free Music Archive \url{http://freemusicarchive.org}
\item pentamusic \url {http://pentamedia.org/pentamusic}
\item Open Clipart Gallery \url{http://openclipart.org}
\item OpenStreetMap \url{http://openstreetmap.de}
\item OpenRouteService \url{http://openrouteservice.org}
\item CC Inhalte finden \url{http://search.creativecommons.org/}
\item CC Inhalte lizensieren \url{https://creativecommons.org/choose/?lang=de}
\end{itemize}


Folie 21:
\begin{itemize}
\item SkoleLinux \url{http://www.slx.no/}
\item LibreOffice \url{https://www.libreoffice.org/} (Weiterentwicklung von \url[OpenOffice]{http://openoffice.org})
\item Firefox, Thunderbird \url{http://mozilla.de/}
\item GIMP \url{http://gimp.org}
\item Inkscape \url{http://inkscape.org/}
\item VLC \url{http://www.videolan.org/vlc/}
\end{itemize}


Folie 22:
\begin{itemize}
\item Datenspuren 2012 \url{http://datenspuren.de}
\end{itemize}
Termine: 12.-14. Oktober 2012, Kulturzentrum Scheune, Dresden


Sonstige Informationen:
\begin{itemize}
\item Luky Luke Comic "Der Betrug": Befasst sich mit verschiedenen Betrugsfallen im Internet. Für Erwachsene und Kinder geeignet. \url{http://economie.fgov.be/de/consommateurs/Betruegereien/LL_betrug/}
\item Sammlung Vorträge CCC Erfa Karlsruhe \url{https://entropia.de/Kategorie:Chaos_macht_Schule}
\item CCC Stuttgart zu Sozialen Netzwerken \url{https://github.com/downloads/Skyr/Vortrag-SocialnetworksGoogleChat/SocialnetworksGoogleChat.pdf}
\item CCC Stuttgart zu Verschlüsselung im Alltag \url{http://skyr.github.com/Vortrag-VerschluesselungImAlltag/folien.html}
\end{itemize}

\end{document}
