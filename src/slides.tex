\documentclass{beamer}
\usetheme{Darmstadt}
\usepackage{graphicx}
\usepackage[german]{babel}
\usepackage[T1]{fontenc}
\usepackage[utf8]{inputenc}

\title{Umgang mit Sozialen Netzwerken}
\author{Chaos Computer Club Dresden}
\date{\today}

\begin{document}
\maketitle

\section{Einleitung}

\subsection{Vorstellung}

\begin{frame}
  \frametitle{Wer sind wir?}
  \begin{itemize}
    \item<1-> Chaos Computer Club Dresden (http://c3d2.de)
      \note{}
    \item<2-> Datenspuren (http://datenspuren.de)
    \item<3-> Podcasts (http://pentamedia.de
    \item<4-> Chaos macht Schule (http://ccc.de/schule)
      \note{}
  \end{itemize}
\end{frame}

\subsection{Anliegen}

\begin{frame}
  \frametitle{Unser Anliegen}
  \begin{itemize}
    \item<1-> Kinder auf das Internet vorbereiten ...
    \item<2-> ... nicht das Internet auf Kinder
      \note{Scheren-Vergleich}
    \item<3-> Medienkompetenz
      \note{Medium nicht nur benutzen, sondern auch verstehen. Wir machen keinen Datenschutz-Richtlinien bei Facebook klicken Vortrag!}
    \item<4-> Kreativer Umgang mit Technik
      \note{Eigene Dinge schaffen, weg von der Konsum-Mentalität}
  \end{itemize}
\end{frame}

\frame{\tableofcontents[hideallsubsections]}

\section{Internet}

\subsection{Grundlagen}

\begin{frame}
  \frametitle{Grundlagen des Internets}
  \begin{itemize}
    \item<1-> Dezentralität
      \note{Es gibt keinen zentralen Server, keinen zentralen Knoten, Pakete nehmen immer andere Routen, Ausfallsicherheit}
    \item<2-> Jeder ist Sender und Empfänger
      \note{Jeder kann Pakete empfangen und senden (nicht nur mit Server, sondern auch untereinander), Demokratie, kein 'Rundfunk' - Nutzer generieren Inhalt (Interaktivität}
    \item<3-> Verlinkung
      \note{Server untereinander Verbunden, Hyperlinks, alles miteinander 'vernetzt', soziale Vernetzung}
    \item<4-> Pseudonymität
      \note{Jeder Rechner im Internet eindeutig identifizierbar, jedoch nicht der Nutzer, keine Anonymität!, Internethund}
  \end{itemize}
\end{frame}

\begin{frame}
  \frametitle{Pseudonymität}
  \begin{figure}
    \includegraphics[height=0.7\textheight]{img/internet_dog.jpg}
    \caption{Copyright Image from New Yorker cartoon by Peter Steiner. http://en.wikipedia.org/wiki/File:Internet\_dog.jpg}
  \end{figure}
\end{frame}

\subsection{Probleme}

\begin{frame}
  \frametitle{Probleme des Internets}
  \begin{itemize}
    \item<1-> Ungeschützte Datenübertragung
      \note{keine Verschlüsselung (erst auf höherer Ebene), jeder auf der Paketroute kann mitlesen}
    \item<2-> Keine Authentifizierung
      \note{TOR-Anonymisierung"` SSL-Verschlüsselung"' Ende-zu-Ende Verschlüsselung (OTR/GPG/PGP für Chat, GPG/PGP für Email)"` Passworteingabe in Formularen, Identitätsdiebstahl}
    \item<3-> "Das Internet vergisst nicht"
      \note{Rechner kopieren Daten!, Privatnutzer, Crawler, Internet-Archive, Screenshot}
  \end{itemize}
\end{frame}

\subsection{Kindernet}

\begin{frame}
  \frametitle{Praxis: Kindernet}
  \begin{figure}
      \includegraphics[height=0.7\textheight]{img/internet_map.jpg}
      \caption{CC-BY https://de.wikipedia.org/wiki/Datei:Internet\_map\_1024.jpg}
      \note{Routing und Ausfallsicherheit, jeder kann Daten lesen, optional: DNS}
  \end{figure}
\end{frame}

\subsection{Historie}

\begin{frame}
  \frametitle{Historie des Internets}
  \begin{itemize}
    \item<1-> Web 1.0
    \item<2-> Zulauf von Firmen und Allgemeinheit
    \item<3-> Web 2.0
  \end{itemize}
\end{frame}

\section{Soziale Netzwerke}

\begin{frame}
  \frametitle{Soziale Netzwerke (1)}
  \begin{figure}
    \includegraphics[height=0.6\textheight]{img/business_pigs.jpg}
    \caption{CC-BY-SA http://geekandpoke.typepad.com/geekandpoke/2010/12/the-free-model.html}
  \end{figure}
\end{frame}

\begin{frame}
  \frametitle{Soziale Netzwerke (2)}
  \begin{itemize}
    \item<1-> Zentralität
    \item<2-> Identität
    \item<3-> Was bedeutet "Befreundet sein"?
    \item<4-> gerichtete/ungerichtete Graphen
  \end{itemize}
\end{frame}

\subsection{Kinderbook}

\begin{frame}
  \frametitle{Praxis}
  Kinderbook
\end{frame}

\section{Freiheit}

\subsection{Back to the Basics}

\begin{frame}
  \frametitle{Back to the Basics}
  \begin{itemize}
    \item<1-> Alle Sender gleichberechtigt
    \item<2-> Dezentrale Dienste
      \begin{itemize}
        \item<3-> Email
        \item<4-> Jabber/XMPP
        \item<5-> Diaspora, Buddycloud
      \end{itemize}
  \end{itemize}
\end{frame}

\subsection{Freie Lizenzen}

\begin{frame}
  \frametitle{Freie Lizenzen}
  \begin{itemize}
    \item<1-> Jeder ist Produzent und Konsument
    \item<2-> Urheberrecht schränkt Verwendung ein
    \item<3-> Freie Lizenzen ermöglichen Verbreitung
    \item<4-> Sharing is caring
  \end{itemize}
\end{frame}

\subsection{Freie Medien}

\begin{frame}
  \frametitle{Freie Medien}
  \begin{itemize}
    \item<1-> Freie Lehrmaterialien
    \item<2-> Freie Musik
      \begin{itemize}
        \item Jamendo
        \item Free Music Archive
        \item Pentamusic
      \end{itemize}
    \item<3-> OpenStreetMap
  \end{itemize}
\end{frame}

\subsection{Freie Software}

\begin{frame}
  \frametitle{Freie Software}
  \begin{itemize}
    \item Windows -> Linux
    \item Microsoft Office -> Libre Office/Open Office
    \item Internet Explorer -> Firefox
    \item Outlook -> Thunderbird
    \item Photoshop -> Gimp
    \item Illustrator -> Inkscape
    \item Windows Mediaplayer -> VLC Media Player
  \end{itemize}
\end{frame}

\section{Fazit}

\subsection{Fazit}

\begin{frame}
  \frametitle{Fazit}
  \begin{itemize}
    \item So what?
  \end{itemize}
\end{frame}

\end{document}
