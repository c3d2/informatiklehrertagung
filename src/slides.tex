\documentclass{beamer}
\usetheme{Darmstadt}
\usepackage{graphicx}
\usepackage[german]{babel}
\usepackage[T1]{fontenc}
\usepackage[utf8]{inputenc}
\setbeamertemplate{footline}[frame number]

\usepackage{pdfcomment}
\newcommand{\ben}[1]{\pdfcomment[author=Ben]{#1}}
\newcommand{\cc}[1]{\includegraphics[height=4mm]{img/#1.png}}
\usepackage{ifthen}
\newcommand{\license}[2][]{\\#2\ifthenelse{\equal{#1}{}}{}{\\\scriptsize\url{#1}}}

\title{Umgang mit Sozialen Netzwerken}
\author{Chaos Computer Club Dresden}
\date{\today}

\begin{document}
\maketitle

\section{Einleitung}
\subsection{}

\begin{frame}
  \frametitle{Wer sind wir?}
  \begin{itemize}
    \item<2-> Chaos Computer Club Dresden (\url{http://c3d2.de})
      \note{}
    \item<3-> Datenspuren (\url{http://datenspuren.de})
    \item<4-> Podcasts (\url{http://pentamedia.de})
    \item<5-> Chaos macht Schule (\url{http://ccc.de/schule})
      \note{alle Folien auf einmal aufblättern? Ben's vorschlag}
  \end{itemize}
\end{frame}

\begin{frame}
  \frametitle{Unsere Anliegen}
  \begin{itemize}
    \item<2-> Kinder auf das Internet vorbereiten \ldots
    \item<3-> \ldots nicht das Internet auf Kinder
      \note{Scheren-Vergleich}
    \item<4-> Informationelle Selbstbestimmung
    \item<5-> Medienkompetenz
      \note{Medium nicht nur benutzen, sondern auch verstehen. Wir machen keinen Datenschutz-Richtlinien bei Facebook klicken Vortrag!}
    \item<6-> Kreativer Umgang mit Technik
      \note{Eigene Dinge schaffen, weg von der Konsum-Mentalität}
  \end{itemize}
	\ben{click-to-speak ratio?}
\end{frame}

\frame{\tableofcontents[hideallsubsections]}

\section{Internet}
\subsection{}

\begin{frame}
  \frametitle{Das Internet}
  \begin{figure}
      \includegraphics[height=0.7\textheight]{img/internet_map.jpg}
			\license[https://de.wikipedia.org/wiki/Datei:Internet\_map\_1024.jpg]{\cc{by}}
			\ben{ich bin mir gerade gar nicht sicher, unter welcher lizenz die cc-icons stehen. Wär natürlich Schrott, wenn die nicht unter Public Domain stehen (dann hätte man ne unendliche Rekursion…)}
      \note{Routing und Ausfallsicherheit, jeder kann Daten lesen, optional: DNS}
  \end{figure}
\end{frame}

\begin{frame}
  \frametitle{Grundlagen des Internets}
  \begin{itemize}
    \item<2-> Dezentralität
      \note{Es gibt keinen zentralen Server, keinen zentralen Knoten, Pakete nehmen immer andere Routen, Ausfallsicherheit}
    \item<3-> Jeder ist Sender und Empfänger
      \note{Jeder kann Pakete empfangen und senden (nicht nur mit Server, sondern auch untereinander), Demokratie, kein 'Rundfunk' - Nutzer generieren Inhalt (Interaktivität}
    \item<4-> Verlinkung
      \note{Server untereinander Verbunden, Hyperlinks, alles miteinander 'vernetzt', soziale Vernetzung}
    \item<5-> Pseudonymität
      \note{Jeder Rechner im Internet eindeutig identifizierbar, jedoch nicht der Nutzer, keine Anonymität!, Internethund}
  \end{itemize}
\end{frame}

\begin{frame}
  \frametitle{Pseudonymität}
  \begin{figure}
    \includegraphics[height=0.7\textheight]{img/internet_dog.jpg}
    \license[http://en.wikipedia.org/wiki/File:Internet_dog.jpg]{\copyright Image from New Yorker cartoon by Peter Steiner.}
  \end{figure}
\end{frame}

\begin{frame}
  \frametitle{Probleme des Internets}
	\ben{Titel besser: Probleme im Internet?}
  \begin{itemize}
    \item<2-> Ungeschützte Datenübertragung
      \note{keine Verschlüsselung (erst auf höherer Ebene), jeder auf der Paketroute kann mitlesen}
    \item<3-> Keine Authentifizierung
      \note{TOR-Anonymisierung "`SSL-Verschlüsselung"' Ende-zu-Ende Verschlüsselung (OTR/GPG/PGP für Chat, GPG/PGP für E-Mail)"` Passworteingabe in Formularen, Identitätsdiebstahl}
    \item<4-> "`Das Internet vergisst nicht"'
      \note{Rechner kopieren Daten!, Privatnutzer, Crawler, Internet-Archive, Screenshot}
  \end{itemize}
	\ben{Ich würde an dieser Stelle statt der ersten beiden Punkten "Vertraulichkeit" und "Integrität" sagen (das ganze noch mit dem CIA Logo schmücken Confidentiality, Integrity and Availability, damit es auch keiner vergisst!). Evtl. könnte man hier auch sagen, dass Kryptographie mittlerweile Bestandteil des Informatiklehrplans ist und man dort auch noch was mit erklären kann. Achtung: TOR gehört IMHO nicht zu Authentifizierung (steht in den notes). Statt "Das Internet vergisst nicht" könnte man auch ein Bild von den Fußstapfen auf dem Mond vs. Fußstapfen im Sand machen (Bilder sind einprägsamer)}
\end{frame}

\begin{frame}
  \frametitle{Praxis: Kindernet}
  \begin{figure}
      \includegraphics[height=0.7\textheight]{img/kindernet.png}
			\license{CC}\ben{was ist CC? Ist damit CC0 gemeint?}
      \note{Routing und Ausfallsicherheit, jeder kann Daten lesen, optional: DNS}
  \end{figure}
\end{frame}

\begin{frame}
  \frametitle{Historie des Internets}
  \begin{itemize}
    \item<2-> Web 1.0
      \note{Jeder eigene 'Homepage', 'Internet of geeks'}
    \item<3-> Zulauf von Firmen und Allgemeinheit
    \item<4-> Web 2.0
      \note{Partizipation auch für 'normale Nutzer', größere Dienste}
    \item<5-> zunehmende Zentralisierung
      \note{'Google ist das Internet und Facebook der einzige Dienst'}
  \end{itemize}
\end{frame}

\section{Soziale Netzwerke}
\subsection{}

\begin{frame}
  \begin{itemize}
  \frametitle{Soziale Netzwerke}
    \item<2-> bieten Nutzern die Möglichkeit sich zu vernetzen
      \note{Unterscheidung: Interessensgebiete, Freundschaftsbeziehung (gerichtete/ungerichtete Graphen)}
    \item<3-> Beispiele
      \note{Frage: Wer ist bei Facebook?}
      \begin{itemize}
        \item<4-> E-Mail, Mailinglisten
        \item<5-> Jabber, Skype, ICQ, MSN
        \item<6-> Facebook, Google+, VZ-Netzwerke, Diaspora
        \item<7-> Twitter, Identi.ca
        \item<8-> Flickr, Picasa
        \item<9-> Github, Bitbucket, Sourceforge
        \item<10-> Foren
      \end{itemize}
  \end{itemize}
\end{frame}

\begin{frame}
  \frametitle{Soziale Netzwerke (1)}
  \begin{itemize}
    \item<2-> Zentralität
    \item<3-> Identität
    \item<4-> Was bedeutet "`Befreundet sein"'?
    \item<5-> gerichtete/ungerichtete Graphen
  \end{itemize}
\end{frame}

\begin{frame}
  \frametitle{Soziale Netzwerke - Geschäftsmodelle}
  \uncover<2->{
  \begin{figure}
    \includegraphics[height=0.6\textheight]{img/business_pigs.jpg}
		\license[http://geekandpoke.typepad.com/geekandpoke/2010/12/the-free-model.html]{\cc{by-sa}}
  \end{figure}
\end{frame}

\begin{frame}
  \frametitle{Praxis: Kinderbook}
  TODO Koeart: Symbolfoto, Fotos Kindernet
\end{frame}

\begin{frame}
  \frametitle{Soziale Netzwerke - Lernziel}
  \begin{itemize}
    \item<2-> Verantungsvoller Umgang mit eigenen Daten
    \item<3-> Verantungsvoller Umgang mit fremden Daten
    \item<4-> Mit wem redet man?
      \note{Ist das wirklich Justin Bieber, der da schreibt?}
    \item<5-> Facebook weiß, was du letzten Sommer getan hast
    \item<6-> "`Für wieviel Geld würdest du deine Daten bei Facebook verkaufen?"'
    \item<7-> Welches Netzwerk für welchen Zweck
      \note{nicht alles muss man über Facebook machen, Verteilung der Daten statt zentral}
  \end{itemize}
\end{frame}

\begin{frame}
  \frametitle{Wie weiter?}
  \begin{itemize}
    \item Einbauen der Ideen in den Unterricht
      \note{Besser noch in die Lehrpläne! Themen: Kinder auf Internet vorbereiten,\ldots nicht das Internet auf die Kinder, keine Internetsperren, Eltern haben Erziehungsauftrag, grundlegendes Verständnis des Internets und seiner Dienste}
    \item Wir stehen zur Verfügung mit Rat und Tat (Aktionstage, Elternabende, Lehrerversammlungen,\ldots)
    \item Junghackertrack bei den Datenspuren
  \end{itemize}
\end{frame}

\section{Freiheit}
\subsection{}

\begin{frame}
  \frametitle{Titel überarbeiten: Was wir vermitteln wollen (?) / Back to the basics}
  \begin{itemize}
    \item<2-> Dezentrale Dienste
      \note{man kann sich die Organisation aussuchen die seine Daten bekommt bzw. einen eigenen Server betreiben}
      \begin{itemize}
        \item<4-> Email
        \item<5-> Jabber/XMPP
        \item<6-> Diaspora, Buddycloud
      \end{itemize}
    \item<3-> Alle Sender gleichberechtigt
    \item<4-> Unix-Philosophie: "`Do one thing, do it right"' (Doug McIlroy)
  \end{itemize}
\end{frame}

\begin{frame}
  \frametitle{Freie Lizenzen}
  \begin{itemize}
    \item<2-> Jeder ist Produzent und Konsument
    \item<3-> Urheberrecht schränkt Verwendung ein
    \item<4-> Freie Lizenzen ermöglichen Verbreitung
			\ben{an dieser Stelle vielleicht CopyLeft als Buzzword erwähnen (daran erinnern sich dann vielleicht die Leute)}
    \item<5-> Sharing is caring
  \end{itemize}
\end{frame}

\begin{frame}
  \frametitle{Freie Medien}
  \begin{itemize}
    \item<2-> Freie Lehrmaterialien
    \item<3-> Freie Musik
      \begin{itemize}
        \item Jamendo
        \item Free Music Archive
        \item Pentamusic
      \end{itemize}
    \item<4-> OpenStreetMap
  \end{itemize}
\end{frame}

\begin{frame}
  \frametitle{Freie Software}
  TODO Beispiele umdrehen!
  \begin{itemize}
    \item Windows -> Linux
    \item Microsoft Office -> Libre Office/Open Office
    \item Internet Explorer -> Firefox
    \item Outlook -> Thunderbird
    \item Photoshop -> Gimp
    \item Illustrator -> Inkscape
    \item Windows Mediaplayer -> VLC Media Player
  \end{itemize}
\end{frame}

\end{document}
