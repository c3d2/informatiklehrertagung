\documentclass{beamer}
\usetheme{Warsaw}
\usepackage[utf8]{inputenc}
\usepackage{graphicx}
 
\title{Umgang mit Sozialen Netzwerken}
\author{Chaos Computer Club Dresden}
\date{\today}
 
\begin{document}
\maketitle
\frame{\tableofcontents[currentsection]}
 
\section{Meta}

\begin{frame}
  \frametitle{Wer sind wir?}
  \begin{itemize}
    \item Chaos Computer Club Dresden
    \item Datenspuren
    \item Podcasts
    \item Chaos macht Schule
  \end{itemize}
\end{frame}

\begin{frame}
  \frametitle{Unser Anliegen}
  \begin{itemize}
    \item Kinder auf das Internet vorbereiten ...
    \item ... nicht das Internet auf Kinder
    \item Medienkompetenz
    \item Kreativer Umgang mit Technik
  \end{itemize}
\end{frame}

\section{Internet}

\begin{frame}
  \frametitle{Grundlagen des Internets}
  \begin{itemize}
    \item Dezentralität
    \item Jeder ist Sender und Empfänger
    \item Verlinkung
    \item Pseudonymität
  \end{itemize}
\end{frame}

\begin{frame}
  \frametitle{Probleme des Internets}
  \begin{itemize}
    \item Ungeschützte Datenübertragung
    \item Keine Authentifizierung
    \item "Das Internet vergisst nicht"
  \end{itemize}
\end{frame}

\begin{frame}
  \frametitle{Praxis}
  Kindernet
\end{frame}

\begin{frame}
  \frametitle{Historie des Internets}
  \begin{itemize}
    \item Web 1.0
    \item Zulauf von Firmen und Allgemeinheit
    \item Web 2.0
  \end{itemize}
\end{frame}

\section{Soziale Netzwerke}

\begin{frame}
  \frametitle{Soziale Netzwerke (1)}
  \begin{figure}
    \includegraphics{img/business_pigs.jpg}
    \caption{CC-BY-SA http://geekandpoke.typepad.com/geekandpoke/2010/12/the-free-model.html}
  \end{figure}
\end{frame}

\begin{frame}
  \frametitle{Soziale Netzwerke (2)}
  \begin{itemize}
    \item Zentralität
    \item Identität
    \item Was bedeutet "Befreundet sein"?
    \item gerichtete/ungerichtete Graphen
  \end{itemize}
\end{frame}

\begin{frame}
  \frametitle{Praxis}
  Kinderbook
\end{frame}

\section{Freiheit}

\begin{frame}
  \frametitle{Back to the Basics}
  \begin{itemize}
    \item Alle Sender gleichberechtigt
    \item Dezentrale Dienste
    \begin{itemize}
      \item Email
      \item Jabber/XMPP
      \item Diaspora, Buddycloud
    \end{itemize}
  \end{itemize}
\end{frame}

\begin{frame}
  \frametitle{Freie Lizenzen (1)}
  \begin{itemize}
    \item Jeder ist Produzent und Konsument
    \item Urheberrecht schränkt Verwendung ein
    \item Freie Lizenzen ermöglichen Verbreitung
    \item Sharing is caring
  \end{itemize}
\end{frame}

\begin{frame}
  \frametitle{Freie Medien}
  \begin{itemize}
    \item Freie Lehrmaterialien
    \item Freie Musik
    \begin{itemize}
      \item Jamendo
      \item Free Music Archive
      \item Pentamusic
    \end{itemize}
    \item OpenStreetMap
  \end{itemize}
\end{frame}

\begin{frame}
  \frametitle{Freie Software}
  \begin{itemize}
    \item Windows -> Linux
    \item Microsoft Office -> Libre Office/Open Office
    \item Internet Explorer -> Firefox
    \item Outlook -> Thunderbird
    \item Photoshop -> Gimp
    \item Illustrator -> Inkscape
    \item Windows Mediaplayer -> VLC Media Player
  \end{itemize}
\end{frame}

\section{Fazit}

\begin{frame}
  \frametitle{Fazit}
  \begin{itemize}
    \item So what?
  \end{itemize}
\end{frame}

\end{document}
